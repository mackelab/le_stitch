\documentclass[10pt,english]{article}
%\special{papersize=210mm,297mm}
\usepackage[a4paper,left=20mm,right=20mm, top=2.5cm, bottom=2.5cm]{geometry}
\usepackage{epsfig}
\usepackage{babel}
\usepackage{amssymb}
\usepackage{amsmath}
%\usepackage{amsthm}
\usepackage{units}
\usepackage[authoryear,round,comma]{natbib}
\bibliographystyle{abbrvnat}
\usepackage{pstricks}
\usepackage{color}

\newcommand{\Fig}{Fig.\ }
\newcommand{\bs}{\backslash}
\renewcommand{\(}{\left(}
\renewcommand{\)}{\right)}
\newcommand{\la}{\left\langle}
\newcommand{\ra}{\right\rangle}

% operators
\newcommand{\Z}{\mathbb{Z}}
\newcommand{\R}{\mathbb{R}}
\newcommand{\I}{\mathbb{I}}
\newcommand{\E}{\mathbb{E}}

\newcommand{\Normal}{\mathcal{N}}
%\newcommand{\Poisson}{\mathcal{P}}  %Poisson likelihood
\newcommand{\Poisson}{\operatorname{Poisson}}  %Poisson likelihood
\newcommand{\GLM}{\operatorname{GLM}}
\newcommand{\Bernoulli}{\operatorname{Bernoulli}}

\newcommand{\diag}{\operatorname{diag}}
\newcommand{\Cov}{\operatorname{Cov}}
\newcommand{\vvec}{\operatornam{vec}}
\newcommand{\abs}{\operatorname{abs}}
\newcommand{\Var}{\operatorname{Var}}
\newcommand{\myvec}{\operatorname{vec}}
\newcommand{\logdet}{\operatorname{logdet}}
\newcommand{\tr}{\operatorname{tr}}
\newcommand{\rank}{\operatorname{rank}}
\newcommand{\argmax}{\operatorname{argmax}}
\newcommand{\argmin}{\operatorname{argmin}}
\newcommand{\compi}{\mathbf{i}}


% bold characters
\newcommand{\xb}{\mathbf{x}}  %x latent state GP
\newcommand{\zb}{\mathbf{z}} 
\newcommand{\rb}{\mathbf{r}} 
\newcommand{\Wb}{\mathbf{W}} 
\newcommand{\ab}{\mathbf{a}} 
\newcommand{\mb}{\mathbf{m}} 
\newcommand{\bb}{\mathbf{b}}
\newcommand{\ub}{\mathbf{u}}
\newcommand{\yb}{\mathbf{y}}  %y observed spike counts
\newcommand{\db}{\mathbf{d}} %d strung out 
\newcommand{\wb}{\mathbf{w}} %d strung out 
\newcommand{\mub}{\bm{\mu}}
\newcommand{\hb}{\mathbf{h}}
  

% SSID stuff
\newcommand{\ybf}{\yb^{+}}
\newcommand{\ybp}{\yb^{-}}
\newcommand{\ybfp}{\yb^{\pm}}
\newcommand{\yfp}{y^{\pm}}
\newcommand{\zbf}{\zb^{+}}
\newcommand{\zbp}{\zb^{-}}
\newcommand{\zbfp}{\zb^{\pm}}
\newcommand{\zfp}{z^{\pm}}


% Matrices
\newenvironment{Mat}[1][0]
{ \left(\begin{array}{#1}}  
{ \end{array}\right) }


% References on the bottom of the page
\usepackage[absolute,overlay]{textpos} 
\newenvironment{reference}[2]{% 
\begin{textblock*}{\textwidth}(#1,#2) 
      \footnotesize\bgroup\color{black}}{\egroup\end{textblock*}}



% Itemization with larger spaces
\newenvironment{myit}[1][0]
{ \begin{itemize}
  \setlength{\itemsep}{5pt}
  \setlength{\parskip}{5pt}
  \setlength{\parsep}{0pt}}
{ \end{itemize} }



% Vertically & horizontally centered tabular entrie
\usepackage{array}
\newcolumntype{L}[1]{>{\raggedright\let\newline\\\arraybackslash\hspace{0pt}}m{#1}}
\newcolumntype{C}[1]{>{\centering\let\newline\\\arraybackslash\hspace{0pt}}m{#1}}
\newcolumntype{R}[1]{>{\raggedleft\let\newline\\\arraybackslash\hspace{0pt}}m{#1}}



\title{Variational Inference for the PLDS}
\author{Lars Buesing}
\date{\today}

\newtheorem{myfact}{Fact}
\newtheorem{myobs}{Observation}
\newtheorem{mydef}{Definition}
\newtheorem{mythm}{Theorem}
\newtheorem{mylem}{Lemma}
\newtheorem{myprob}{Problem}
\newtheorem{mycol}{Corollary}

  

\begin{document}
\maketitle

\section{Model}

\begin{eqnarray}
 p(\xb)				&=&	\Normal(\xb\vert \mu,\Sigma)\\
 p(\yb\vert \xb)		&=&	\prod_{n}	p(y_n\vert \eta_n), \ \ \ \ \eta:=W\xb+\overline{\db}
\end{eqnarray}
For latent dynamical system we know that $\Lambda:=\Sigma^{-1}$ is tri-diagonal, and $W$ block-diagonal:
\begin{eqnarray}
 \Lambda	&=&	\begin{pmatrix} Q_0^{-1}+AQ^{-1}A^\top	&	A^\top Q^{-1}		& \\
					Q^{-1}A			&	Q^{-1}+AQ^{-1}A^\top 	&  A^\top Q^{-1}\\
								&	\ddots			& \ddots		& \ddots
        	   	\end{pmatrix}\\
 W		&=&	\operatorname{blk-diag}(\underbrace{C,\ldots,C}_{T\mbox{-times}})
\end{eqnarray}



\section{Inference problem}

Gaussian variational approximation:
\begin{eqnarray}
 q(\xb)				&=&	\Normal(\xb\vert\mb,V)
\end{eqnarray}
Variational lower bound:
\begin{eqnarray}
 \mathcal L(\mb,V)		&\leq& \log p(\yb)\\
 \mathcal L(\mb,V)		&=&	\frac{1}{2}\left(\log\vert V\vert-\tr[\Sigma^{-1}V]-(\mb-\mu)^\top \Sigma^{-1}(\mb-\mu)     \right)
					+\sum_{n} \E_{q(\xb)}[\log p(y_n\vert \eta_n)]
\end{eqnarray}
For Poisson with exp link function we can compute $\E_{q(\xb)}[\log p(y_n\vert \eta_n)]$, otherwise (eg for Bernoulli observations) use a local variational lower bound on the integrated likelihood:
\begin{eqnarray}
 \E_{q(\xb)}[\log p(y_n\vert \eta_n)]	&\geq& 	-f_n(\overline m_n,\overline v_n)\\
 f_n(\overline m_n,\overline v_n)	&=&	-y_n\overline m_n+\exp(\overline m_n+\overline v_n/2)\\
 \overline \mb				&:=&	W\mb+\overline\db\\
 \overline{ \mathbf{v}}			&:=&	\diag(WVW^\top)
\end{eqnarray}
The bound then reads:
\begin{eqnarray}
 \mathcal L(\mb,V)	&=&	\frac{1}{2}\left(\log\vert V\vert-\tr[\Sigma^{-1}V]-(\mb-\mu)^\top \Sigma^{-1}(\mb-\mu)     \right)
				-\sum_{n} f_n(\overline m_n,\overline v_n)
%\\			&=&	\frac{1}{2}\left(\log\vert V\vert-\tr[\Sigma^{-1}V]-\Vert\mb-\mu\Vert^2_{\Sigma^{-1}}    \right)-\sum_{n} f_n(\overline m_n,\overline v_n)
\end{eqnarray}
For convex $f_n$ (true for exp-PLDS): strictly concave optimization in $\mb,V$\\
Possible optimization strategies:
\begin{enumerate}
 \item Direct optimization over $\mb,V$: strictly concave, however $V$ dense; does not make use of Markovian structure of the model 
 \item Optimization over $\mb,V^{-1}$: Opper et al show that optimal $V^\ast=(\Sigma^{-1}+W^\top \diag(\lambda)W)^{-1}$; hence for tri-diagonal $\Sigma^{-1}$ and block-diagonal $W$ then
       $V^\ast$ is also tri-diagonal; however optimization over $\mb,\lambda$ is not convex and converges slowly according to [Seeger et al. ICML2013]
 \item Solve the dual optimization as proposed in [Seeger et al. ICML2013]: convex, makes use of Markovian structure of the model 
\end{enumerate}


\section{Variational inference via dual optimization}
\subsection{Optimization to solve}
Dual problem:
\begin{eqnarray}
 \min_{\lambda}		&&	\frac{1}{2} (\lambda-\yb)^\top W\Sigma W^\top(\lambda-\yb)-(W\mu+\overline{\db})^\top(\lambda-\yb)-\frac{1}{2}\log\vert A_\lambda\vert+\sum_nf^\ast(\lambda_n)\\
 \mbox{subject to}	&&	\lambda_i>0 \nonumber
\end{eqnarray}
where 
\begin{eqnarray}
 f^\ast(\lambda_i)	&:=&		\lambda_i(\log\lambda_i-1)\\
 A_\lambda		&:=&		\Sigma^{-1}+W^\top \diag(\lambda)W
\end{eqnarray}
The optimal variational parameters for $q(\xb)=\Normal(\xb\vert\mb^\ast,V^\ast)$ are given by:
\begin{eqnarray}
 \mb^\ast		&=&		\mu-\Sigma W^\top(\lambda^\ast-\yb)\\
 V^\ast			&=&		(\Sigma^{-1}+W^\top \diag(\lambda^\ast)W)^{-1}=A_{\lambda^\ast}^{-1}
\end{eqnarray}
\subsection{How to optimize?}
Use gradient based methods:
\begin{eqnarray*}
 \nabla_\lambda		&=&		W\Sigma W^\top (\lambda-\yb)-W\mu-\overline{\db}+\log\lambda-\frac{1}{2}\diag(WA^{-1}_\lambda W^\top)\\
			&=&		\underbrace{W\Sigma W \lambda}_{O(N)}+\underbrace{\log\lambda}_{O(N)}-\frac{1}{2}\diag(\underbrace{WA^{-1}_\lambda W^\top}_{O(N)})-\underbrace{W(\Sigma W^\top\yb+\mu)}_{\mbox{pre-compute}}
\end{eqnarray*}
Hessian:
\begin{eqnarray*}
 H_\lambda		&=&		\diag(\lambda)^{-1}+W\Sigma W^\top + (WA^{-1}_\lambda W^\top)\circ(WA^{-1}_\lambda W^\top)
\end{eqnarray*}
Iterate:
\begin{eqnarray*}
 \mb^k			&=&		\mu+\Sigma W^\top\yb-\Sigma W^\top\lambda^k\\
 A^k			&=&		\Sigma^{-1}+W^\top \diag(\lambda^k)W\\
 \nabla^k		&=&		\log\lambda^k-W\mb^k-\overline{\db}-\frac{1}{2}\diag(W(A^k)^{-1}W^\top)\\
 \lambda^{k+1}		&=&		\lambda^{k}-\nu \nabla^k
\end{eqnarray*}
Computing the block-diagonal elements of $A^k$ is equivalent to Kalman smoothing and requires a forward-backward pass through the data which costs $O(Td^3)$.\\
What's the relation to Laplace approximation?
\begin{eqnarray*}
 \nabla^k		&=&		-\Sigma^{-1}(\xb-\mu)+W^\top(\yb-\exp(W\xb+\overline{\db})) \\
 H^k			&=&		-(\Sigma^{-1}+W^\top \diag(\exp(W\xb+\overline{\db}))W)
\end{eqnarray*}


\subsection{Kalman smoothing}
The matrix $A_\lambda$ equals exactly the precision matrix of a LDS with dynamics given by $A,Q$ and observations sampled from $\Normal(C\xb_t,\diag(\lambda_t))$.
Hence, calculating the block-diagonal of $A^{-1}_\lambda$ is exactly equivalent to calculating the smoothed posterior covariance of this LDS.
Let $P_{t\vert t}$ denote the filtered covariance, $P_{t+1\vert t}$ the one-step-ahead covariance and $P_{t\vert T}$ the smoothed covariance of this model.
\begin{eqnarray}
 (A^{-1}_\lambda)_{(t-1)d+1:td,(t-1)d+1:td}	&\stackrel{!}{=}&	P_{t\vert T}
\end{eqnarray}
We use the Kalman smoother recursions. The forward pass reads:
\begin{eqnarray}
 P_{t+1\vert t}		&=&	AP_{t\vert t}A^\top+Q\\
 P_{t+1\vert t+1}	&=&	\(P_{t+1\vert t}^{-1}+C^\top\diag(\lambda_t)C\)^{-1}\\
			&=&	\(I+P_{t+1\vert t}C^\top\diag(\lambda_t)C\)\backslash P_{t+1\vert t}\\
 P_{0\vert 0}		&=&	Q_0
\end{eqnarray}
The backward pass is given by:
\begin{eqnarray}
 C_t			&=&	P_{t\vert t}A^\top/P_{t+1\vert t}\\
 P_{t\vert T}		&=&	P_{t\vert t}+C_t\( P_{t+1\vert T}- P_{t+1\vert t}\)C_t^\top
\end{eqnarray}
The initialization for the backward pass $P_{T\vert T}$ is calculated the last step of the forward pass.




\clearpage
\section{Appendix}
\subsection{Derivation of dual optimization}
Original primal problem:
\begin{eqnarray*}
\max_{\mb,V}	&	\frac{1}{2}\left(\log\vert V\vert-\tr[\Sigma^{-1}V]-\Vert\mb-\mu\Vert^2_{\Sigma^{-1}}    \right)-\sum_{n} f_n(\overline m_n,\overline v_n)\\
\mbox{s.t.}	&	V\in S^{++}
\end{eqnarray*}
Expanded primal problem:
\begin{eqnarray*}
 \begin{array}{ll} 	\argmax_{\mb,V,\rho,h}	&	\frac{1}{2}\left(\log\vert V\vert-\tr[\Sigma^{-1}V]-\Vert\mb-\mu\Vert^2_{\Sigma^{-1}}    \right)-\sum_{n} f_n(h_n,\rho_n)\\
			\mbox{s.t.}		&	V\in S^{++}\\
						&	h=W\mb+\overline{\db}\\
						&	\rho=\diag(WVW^\top)
 \end{array}
\end{eqnarray*}
Lagrangian:
\begin{eqnarray*}
  \mathcal L(\mb,V,h,\rho,\alpha,\lambda)	&:=&	\frac{1}{2}\left(\log\vert V\vert-\tr[\Sigma^{-1}V]-\Vert\mb-\mu\Vert^2_{\Sigma^{-1}}    \right)-\sum_{n} f_n(h_n,\rho_n)\\
						&&	+\alpha^\top(h-W\mb+\overline{\db})+\frac{1}{2}\lambda^\top(\rho-\diag(WVW^\top))
\end{eqnarray*}
Dual
\begin{eqnarray*}
 D(\alpha,\lambda)				&:=&	\min_{\mb,V,h,\rho}L(\mb,V,h,\rho,\alpha,\lambda)\\
 V^\ast						&=&	(\Sigma^{-1}+W^\top \diag(\lambda)W)^{-1}=:A_\lambda^{-1}\\
\mb^\ast					&=&	\mu-\Sigma W^\top\alpha\\
\alpha						&=&	\lambda-y
\end{eqnarray*}
Final reduced dual problem:
\begin{eqnarray*}
 \begin{array}{ll} 	\argmin_{\lambda}	&	\frac{1}{2} (\lambda-\yb)^\top W\Sigma W(\lambda-\yb)-(\overline\db+W\mu)^\top(\lambda-\yb)-\frac{1}{2}\log\vert A_\lambda\vert+\sum_nf^\ast(\lambda_n)\\
			\mbox{s.t.}		&	\lambda_i>0
 \end{array}
\end{eqnarray*}


\subsection{Duality basics}
Primal problem with optimal values $p^\ast$:
\begin{eqnarray*}
 \begin{array}{ll} 	\mbox{min}	&	f(x)\\
			\mbox{s.t.}	&	f_i(x)\leq 0\\
					&	h_i(x)=0
 \end{array}
\end{eqnarray*}
Lagrange function with $\lambda_i\geq0$:
\begin{eqnarray*}
 L(x,\lambda,\nu)	&:=&	f(x)+\sum_{i}\lambda_if_i(x)+\sum_i\nu_ih_i(x)
\end{eqnarray*}
Dual:
\begin{eqnarray*}
 g(\lambda,\nu)	&:=&	\inf_x L(x,\lambda,\nu)
\end{eqnarray*}
Dual is a lower bound:
\begin{eqnarray*}
 g(\lambda,\nu)	&\leq&	p^\ast
\end{eqnarray*}
This can be shown by bounding the constraint functions with linear functions from below.
Dual problem:
\begin{eqnarray*}
 \begin{array}{ll} 	\mbox{min}	&	-g(\lambda,\nu)\\
			\mbox{s.t.}	&	\lambda_i\geq 0
 \end{array}
\end{eqnarray*}
Dual is always convex!
\paragraph{Slater's conditions}
We have stong duality iff:
\begin{eqnarray*}
 g(\lambda,\nu)	&=&	p^\ast
\end{eqnarray*}
A sufficient condition for strong duality is: $f$ convex, no inequality constraints, primal feasible

\paragraph{Dual function}
The dual $f^\ast$ of a function $f$ is defined as:
\begin{eqnarray*}
 f^\ast(y)	&:=&	\sup_x \   y^\top x - f(x)
\end{eqnarray*}



\end{document}
